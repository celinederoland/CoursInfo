\titre{Principe du chiffrement asymétrique :}
\begin{itemize}
	\item Je donne ma clé publique
	\item L'autre donne sa clé publique.
	\item Je chiffre avec la clé publique du destinataire et il déchiffre avec sa clé privée
	\item Le destinataire chiffre avec ma clé publique et je déchiffre avec ma clé privée
\end{itemize}

\titre{Différences :}\\
\begin{tabular}{p{0.5\linewidth}|p{0.5\linewidth}}
	Symétrique & Asymétrique \\ \hline \hline
	plus rapide (la clé est sur 128bits) & lent (la clé est sur 2048bits) \\ \hline
	phase critique : échange de la clé & échange des clés facilités \\ \hline
	la confidentialité est assurée & il faut s'assurer de l'identité de l'émetteur de la clé publique \\ \hline
\end{tabular}

\titre{Code chiffré :} On dit qu'il est pseudo-aléatoire :
\begin{itemize}
	\item pseudo : ce n'est pas aléatoire
	\item aléatoire : ça a l'air aléatoire (on ne peut pas reconnaitre des motifs)
\end{itemize}

\titre{DES :} Data Encryption Service \\

\titre{AES :} Advanced Encryption Service \\

\titre{Problématique du changement d'algorithme :} Ce sont des puces qui réalisent le cryptage, donc il faut du matériel équipé de ces puces. \\

\titre{Pièges :} Attention : il ne faut pas chiffrer par caractère, car il est alors facile de trouver la correspondance 'caractère' / 'caractère chiffré'.\\

\titre{CBC :} Cipher Bloc Chaining

\titre{Exemple :} Chiffrement d'une connexion WIFI. \\
\begin{tabular}{l|l|l|l|l}
	@MAC-D & @MAC-Src & Ethertype & VI & \ldots zone chiffrée \ldots \\
\end{tabular} \\
Côté récepteur, on prend la clé et on la concatène avec le VI reçu et on génère le même R que côté émetteur. \\
On fait un XOR entre le R et le message reçu, on obtient le message en clair. \\
Il faut connaitre la clé pour générer le R, et ensuite faire le XOR. \\
Le pirate connait le VI mais pas la clé, donc il ne peut pas déchiffrer. \\
Le R est toujours différent donc le pirate ne peut pas reconnaitre des motifs en se basant sur le format classique des trames. \\

\titre{Faille principale du WEP :} Le VI est sur 24 bits, donc environ 16 millions de possibilités. On a à peu près 20 Mbits/s de débit utile sur le réseau, la taille moyenne d'une trame est 1000 octets = 8000 bits. \\
Combien de trames par secondes ? $20.10^6 / 8.10^3 = 2 500$ \\
Au bout de combien de temps les VI rebouclent ? $16.10^6 / 2 500 = 6400s = 1h45$\\
Comme ça reboucle fréquemment, il est impossible d'empêcher de rejouer les VI. \\
Exemple : Le pirate va capturer une trame ARP avec un VI donné. Il capture C = M XOR R. Il devine M car il connait les trames ARP. Il calcule R = C XOR M. Il connait $R_{VI_1}$. Il émet des trames en utilisant ce $R_{VI_1}$. Lorsque le point d'accès répond avec un autre $VI$, or le pirate connait la réponse donc il peut calculer $R_{VI_2}$. A la fin il peut posséder une table de correspondance pour chaque $VI$ (sans avoir eu besoin de découvrir la clé).
