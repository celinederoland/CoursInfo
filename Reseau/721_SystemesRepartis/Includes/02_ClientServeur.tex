\titre{Le client} Effectue une demande de service (une requête). C'est lui qui initie le contact.\\
\titre{Le serveur} Offre le service, répond aux requêtes. \\
\titre{Système coopératif} Les parties clients et serveurs peuvent s'exécuter sur des systèmes différents. Un serveur peut répondre à plusieurs clients. \\
\titre{Problème de temps} principalement pour la voix et la visioconférence \\
\titre{Couches concernées} applicative, traitement(= logique), données. \\
\titre{Questions :} Comment répartir le découpage d'une application informatique de type clients / serveurs ?
\titre{Modèle de Gartner (2 tiers)} \\
Client :
\begin{tabular}{|c|}
Présentation \\ \hline Logique \\ \hline Données \\ \hline
\end{tabular}
\begin{tabular}{|c|}
Présentation \\ \hline Logique \\ \hline
\end{tabular}
\begin{tabular}{|c|}
Présentation \\ \hline Logique \\ \hline
\end{tabular}
\begin{tabular}{|c|}
Présentation \\ \hline 
\end{tabular}
\begin{tabular}{|c|}
Présentation \\ \hline
\end{tabular}
\\
Serveur :
\begin{tabular}{|c|}
Données \\ \hline
\end{tabular}
\begin{tabular}{|c|}
Données \\ \hline
\end{tabular}
\begin{tabular}{|c|}
Données \\ \hline Logique \\ \hline
\end{tabular}
\begin{tabular}{|c|}
Données \\ \hline Logique \\ \hline
\end{tabular}
\begin{tabular}{|c|}
Données \\ \hline Logique \\ \hline Présentation \\ \hline
\end{tabular}
\\
\titre{Communication en mode non connecté : TCP} Le client émet une requête, le serveur se réveille et répond. \\
\titre{Communication en mode connecté : UDP} Le client fait une demande de connexion, le serveur crée un contexte, le client envoie des requêtes et le serveur répond.\\
\titre{Serveurs itératifs ou concurrents}
\begin{itemize}
	\item Itératif : traite séquentiellement les requêtes (souvent en mode non connecté)
	\item Concurrent : les serveur accepte les requêtes puis les délègue à un processus fils (souvent en mode connecté)
\end{itemize}
\titre{Notion d'état}
\begin{itemize}
	\item Service avec états (mode connecté). Le serveur conserve localement un état pour chacun des clients connectés
	\item Service sans état : ne conserve aucune info sur l'enchaînement des requêtes
	\item Sans état est plus performant, Avec état est plus résistant aux pannes
\end{itemize}
\titre{Mémoire partagée}
\begin{itemize}
	\item Communications locales : processus sur une même machine
	\item Communication distante : la mémoire partagée est physiquement répartie.
\end{itemize}
\titre{Communication par passage de messages}
\begin{itemize}
	\item Les processus n'ont pas accès à des variables communes
	\item Ils communiquent en s'échangeant des messages : 2 primitives : send et rcv
	\item ...
	\item Eviter les écritures concurrentes
	\item Opérations bloquantes / non bloquantes :
		\begin{itemize}
			\item rcv peut rendre la main quand les données ont été reçues et recopiées depuis le tampon de réception local.
			\item send peut rendre la main : aussitôt (ou plus tard par sécurité : attente de la réception ou de l'enregistrement ou de la consommation des données par le receveur).
		\end{itemize}
\end{itemize}

\titre{Interface de programmation} \\

\titre{Sockets : principes généraux} canal de com entre deux programmes ou processus. Echange peut se faire dans les deux sens une fois le canal de com en place. Au dessus de la couche transport.\\

\titre{4 types }
\begin{itemize}
	\item Stream
	\item Datagram
	\item Raw
	\item Sequenced packet
\end{itemize}

\titre{Exercice : à quoi correspondent ces primitives ?}
\begin{itemize}
	\item socket() : création
	\item close() : fermeture
	\item accept() : accepter la connection (en mode connecté)
	\item send() : envoie (en mode connecté)
	\item bind() : relier le socket à un port
	\item recvfrom() : recevoir (en mode non connecté)
	\item connect() : (en mode connecté : TCP)
	\item receive() : recevoir (en mode connecté)
	\item listen() : écouter sur un port (serveur)
	\item sendto() : envoie (en mode non connecté)
\end{itemize}

