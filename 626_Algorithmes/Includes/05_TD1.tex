\titre{Exo 1 :} Soit la fonction : \\
\code{
void tri3(int* V, int i, int j) \\
\{ \\
	if (V[i] > V[j])\\
	\{\\
		int tmp = V[i]; V[i] = V[j]; V[j] = tmp;\\
	\}\\
\\
	if (i+1 $\geq$ j) return;\\
	int k = (j-i+1)/3;\\
	tri3(T,i,j-k);\\
	tri3(T,i+k,j);\\
	tri3(T,i,j-k);\\
\}\\
}
\begin{enumerate}
	\item Montrer qu'il trie correctement le tableau
	\item Temps d'exécution asymptotique de tri3
	\item Est il meilleur que trifusion et autres
\end{enumerate}

\begin{enumerate}
	\item 
	\begin{itemize}
		\item Montrons que tri3 fonctionne pour $n\leq 3$
		\begin{itemize}
			\item $n=1$ le tableau est déjà trié
			\item $n=2$ Si il n'est pas trié, l'échange se fait, sinon rien, et ça sort
			\item $n=3$ $i=0,j=2,k=1$. Au premier tour on tri les 2 premières valeurs, puis les 2 dernières, et encore les 2 premières. Donc c'est bon.
		\end{itemize}
		\item Montrons que tri3 fonctionne jusqu'à $n$ alors tri3 fonctionne pour $n+1$
		\begin{itemize}
			\item Simplifions en supposant $n+1$ multiple de $k$
			\item Au premier stade, la zone $i .. j-k$ est triée ($V(l)<V(m)$), donc sur $i+k .. j-k$ on a des valeurs supérieures aux valeurs dans $i .. i+k$ 
			\item Au second stade, la zone $i+k .. j$ est triée ($V(l)<V(m)$), donc les valeurs dans $j-k .. j$ sont supérieures aux valeurs dans $i+k .. j-k$ et dans $i .. i+k$ et elles sont triées.
			\item On trie encore le début du tableau.
		\end{itemize}
	\end{itemize}
	\item Complexité asymptotique de tri3 ? 
	\begin{itemize}
		\item $n = j - i + 1$
		\item $T(n) = \theta(1) + 3T(\frac{2}{3}n)$
		\item Cas 1 du théorème avec $b = \frac{3}{2}$
		\item Donc $T(n) = \theta(n^{\frac{\ln 3}{\ln 3 - \ln 2}}) = \theta(n^{2,71})$
	\end{itemize}
\end{enumerate}

