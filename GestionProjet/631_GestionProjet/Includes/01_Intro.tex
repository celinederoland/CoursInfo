\titre{Qu'est-ce qu'un projet ?}
\begin{itemize}
	\item Satisfaction d'un besoin spécifique et particulier
	\item Se déroule dans un contexte précis
	\item Unique dans le temps et l'espace (caractère novateur, au moins partiel)
	\item Daté, marqué par un début et une fin
	\item Autonome : il se suffit à lui même
	\item Mobilise des ressources d'une façon ponctuelle
	\item Aboutit à un résultat : produit ou service
\end{itemize}

\titre{Principales étapes dans un projet}
\begin{itemize}
	\item Recueil et analyse des besoins 
		\begin{itemize}
			\item interview (questions ouvertes et fermées), recherche documentaire
			\item Analyse des besoins (synthèse, cohérence, complexité, faisabilité)
			\item Evaluation des problèmes techniques, recherche de solutions et d'informations complémentaires
			\item Diviser pour réduire la complexité
			\item Définir une stratégie de réalisation (priorités, organisation)
			\item Etude de faisabilité
			\item Finalisation : Cahier des charges fonctionnel, dossier des besoins et backlog produit
		\end{itemize}
	\item Construction de la solution
		\begin{itemize}
			\item Conception de logiciel
			\item Implémentation et tests
			\begin{itemize}
				\item Spécifications des briques de base
				\item Codage des différents éléments et tests unitaires
				\item Liens entre les briques
				\item Tests de non régression
			\end{itemize}
			\item Documentation technique (dont les commentaires du code)
			\item Dossier de conception
			\item Dossier de spécification
			\item Dossier de tests
		\end{itemize}
	\item Livraison, évolution et maintenance
	\begin{itemize}
		\item Produit finalisé
		\item Documentation complète
		\item Elements utiles pour évolution et maintenance
		\item Bilans et retour d'expérience
		\item Présentation orale finale
	\end{itemize}
\end{itemize}

\titre{Taches :} Les étapes s'enchainent et peuvent se répéter en fonction des choix méthodologiques retenus. Elles se découpent en tâches de plusieurs types
\begin{itemize}
	\item Uniques
	\item Répétitives
	\item Séquentielles
	\item Parallèles
	\item Collaboratives
	\item \ldots
\end{itemize}
Elles doivent être 
\begin{itemize}
	\item Nommées
	\item Evaluées (temps, ressources, coût)
	\item Organisées (dépendances, priorités, \ldots)
	\item Attribuées et réalisées
\end{itemize}
