\titre{Contenu du cahier des charges :} Dire \titre{à quoi va servir} le produit et non ce qu'il sera. On identifie les fonctions qu'il doit remplir pour rendre service à l'utilisateur auquel il est destiné. \\

\titre{Objectif :} Trouver le bon produit (qualité + réponse au besoin) (80 \% des logiciels sont à la poubelle au bout d'un an, 60 \% sont à la poubelle immédiatement)\\

\titre{Analyse fonctionnelle :} 
\begin{itemize}
	\item Traduction des besoins en fonctions à assurer et en tests d'acceptation.
	\item Critères d'appréciation des fonctions (faire valoriser devant le client les tâches chères mais nécessaires)
	\item Prioriser les fonctionnalités proposées en fonction de :
	\begin{itemize}
		\item Positionnement dans l'architecture
		\item Demande du client
		\item Ressources disponibles
		\item Valeur ajoutée
		\begin{itemize}
			\item Pour l'utilisateur (ses attentes, sa motivation, ses capacités) $\rightarrow$ valeur d'usage et valeur d'estime.
			\item Pour le concepteur (complexité, temps et ressources nécessaires)
		\end{itemize}
	\end{itemize}
\end{itemize}

\titre{Contenu :} On proposose une (ou des) réponse(s) aux besoins
\begin{itemize}
	\item Architecture logicielle
	\item Fonctionnalités fournies
	\item Tests proposés
	\item Planning de réalisation et de livraison
\end{itemize}

\titre{Contrat :} Accord sur le produit et le service attendu/réalisé ayant pour objet l'établissement d'obligations à la charge ou au bénéfice de chacune de ses parties. \\

\titre{Propriétés d'un bon cahier des charges :}
\begin{itemize}
	\item Cohérent
	\item Non Ambigu
	\item Précis
	\item Complet
	\item Réalisable, testable et maintenable
	\item Titre, Date, Version, Auteur(s), Liste de diffusion, Mots clés associés, Table des matières et des figures
\end{itemize}

\titre{Plan classique}
\begin{enumerate}
	\item Introduction : présentation du projet et docs de référence
	\item Objectif : description de la future solution (fonctionnalités visibles et invisibles, besoins,technos)
	\item Performance et configuration : prestations attendues
	\item Organisation : intervenants et rôles, répartition des activités, suivi
	\item Capacité et qualité de service (noms de variable, commentaires, tests)
	\item Calendrier
	\item Glossaire, lexique
\end{enumerate}
