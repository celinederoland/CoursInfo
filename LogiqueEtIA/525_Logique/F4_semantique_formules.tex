\titre{Domaine sémantique} (domaine d'interprétation, domaine de discours) : $\{0,1\}$.
\titre{Interpréter une fbf} consiste à lui attribuer une valeur de $\{0,1\}$\\

\titre{Une assignation} sur $n$ propositions est un ensemble d'interprétations de ces propositions (un $n-$uplet). Une assignation = un monde possible = une ligne de la table de vérité (pour $n$ proposition il y a $2^n$ assignations possibles et $2^{2^n}$ fonctions de vérité possibles). \\

\titre{Un modèle} pour une fbf donnée est une assignation qui la rend vraie. \\

\titre{Méthode} Pour déterminer la sémantique d'une formule, on dresse sa table de vérité. \\

\titre{Théorème de substitution} Si une formule est valide (resp. inconsistante), la formule obtenue en substituant chaque occurence d'un schéma de formule par une fbf quelconque est également valide (resp. inconsistante).\\

\titre{Métalangage} $\conslogiquedble$ A signifie A est valide.


