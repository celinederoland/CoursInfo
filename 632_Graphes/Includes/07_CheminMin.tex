\titre{Calcul des CLM : Chemins de Longueur Minimale}
\begin{enumerate}
	\item D'un sommet à un autre (on fait le 2 et on extrait le bon chemin)
	\item D'un sommet à tous les autres (Dijkstra)
	\item De tous les sommets à un autre (On inverse les arêtes du graphe et on fait 2)
	\item De tous les sommets vers tous les sommets (Floyd-Warshall)
\end{enumerate}

\titre{Définition :} Soit $G=(S,A)$ un graphe et $p$ sa fonction de pondération. Un \titre{Chemin de Longueur Minimale} du sommet $u$ au sommet $v$ est un chemin de $[s_0,\ldots,s_k]$ avec $s_0=u$, $s_k=v$ et $\sum p(s_i,s_{i+1})$ minimal.\\

\titre{Arbre de CLM :} Un arbre couvrant enraciné en un sommet $s$ tel que l'unique chemin de $s$ à un sommet $u$ de cet arbre est un CLM du graphe de départ. \\

\titre{Remarques :} 
\begin{itemize} 
	\item L'arbre de CLM existe.
	\item Soit $[s_0,\ldots,s_k]$ un CLM de $s_0$ à $s_k$. Alors pour tout $i\in\{1,\ldots,k-1\}$ on a $[s_0,\ldots s_i]$ est un CLM de $s_0$ à $s_i$
\end{itemize}

\titre{Algorithme de Dijkstra :} Meme principe que l'algo de Prim à la différence que cle[u] est la longueur du plus court chemin connu du sommet de départ au sommet u.\\

\titre{Algorithme de Floyd-Warshall :} L'idée est de construire dynamiquement une suite de matrices $M_0,\ldots,M_n$ telles que $m_{k_{i,j}}$ est le cout du plus court chemine de $i$ à $j$ en utilisant comme sommets intermédiaires les sommets $1,2,\ldots ,k$
