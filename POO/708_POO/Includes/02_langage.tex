\titre{Conventions d'écriture}
\begin{itemize}
	\item Les noms de classe et les variables globales commencent par une majuscule (cf Smalltalk inspect. en eyetreeInspector pour voir la liste de toutes les classes).
	\item Les noms des variables d'instance et des méthodes commencent par une minuscule.
	\item les caractères sont préfixés par \$
	\item Les chaînes de caractère sont entre simple quotes
	\item Les commentaires sont entre guillemets
	\item Les tableau \#(el1 el2 \ldots)
	\item Les symboles sont préfixés par \#
\end{itemize}

\titre{Concepts de base}
\begin{itemize}
	\item Tout objet est instance unique d'une classe (pas de prototype : il faut créer la classe en premier, et ensuite on ne peut pas changer de classe).
	\item Encapsulation des données : les variables d'instance sont protected.
	\item Les variables sont non typés (on cherche la méthode d'après la classe de l'objet référencé par la variable et non d'après le type de la variable).
	\item Liaison purement dynamique.
	\item self = pseudo variable désignant le receveur du message, utilisable uniquement dans les méthodes et obligatoire pour appeler une autre méthode de la même classe.
	\item super = pseudo variable désignant le receveur du message, utilisable uniquement dans les méthodes, indiquant que la recherche de la méthode doit se faire à partir de la surclasse de la classe qui implémente la méthode.
	\item Envoi de message < objet receveur > < message >. Tout envoi de message retourne un objet (l'objet receveur si pas précisé dans la définition de la méthode).
	\item Il n'y a pas de structure de contrôle (uniquement des envois de messages)
	\item Comportement différent $\impl$ Classes différentes
\end{itemize}

\titre{3 types de messages}
\begin{itemize}
	\item unaire sans paramètre
	\item binaire 1 ou 2 caractères non alphanumériques, 1 seul argument
	\item "keywords" au moins 1 argument
\end{itemize}

\titre{Priorité :} Parenthèses, unaires, binaires, keywords

\titre{Services de contrôle :}
\begin{itemize}
	\item Conditionnelle : cas particulier de ifClasse
	\item Itérations :
		\begin{itemize}
			\item whileTrue, whileFalse : \\
				whileTrue: aBlock 
				\\self value
				\\ifTrue: [ 
				\\aBlock value. 
				\\self whileTrue: aBlock ] 
			\item 1 to: 10 do: [:i | x := x + i]
		\end{itemize}
\end{itemize}
