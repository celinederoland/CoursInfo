Soit $X$ un ensemble \\
\titre{Vision ensembliste} : $\Gamma \incluseq X^2$\\
\titre{Vision fonctionnelle} : $\ffonc{\Gamma}{P(X)}$\\
Le couple $(X,\Gamma)$ est appelé graphe sur $X$.\\
$X$ est l'ensemble des \titre{sommets}.\\
$\Gamma$ est l'ensemble des \titre{arcs} ou la \titre{fonction d'incidence}.\\

\vskip 0.6cm

\titre{Relation entre deux sommets} :
Si $(x,y)\in\Gamma$ ou $y\in\Gamma(x)$, on note  \titre{$x\rel y$} et on dit que 
\begin{itemize}
	\item $x$ est l'\titre{origine} ou le \titre{prédécesseur} de $y$. On note \titre{$\Gamma^{-1}(y)$} l'ensemble des prédécesseurs de $y$.
	\item $y$ est l'\titre{extrémité} ou le \titre{successeur} de $x$. On note \titre{$\Gamma(x)$} l'ensemble des successeurs de $x$.
\end{itemize}

\vskip 0.8cm

\titre{Degré des sommets}
\begin{itemize}
	\item $d^+(x)=|\Gamma(x)|$ \titre{$\frac{1}{2}$ degré extérieur}
	\item $d^-(x)=|\Gamma^{-1}(x)|$ \titre{$\frac{1}{2}$ degré intérieur}
	\item $d(x)=d^+(x)+d^-(x)$ \titre{degré}
\end{itemize}
