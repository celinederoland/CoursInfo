\titre{Théorie de la calculabilité} : Cactérisation des fonctions calculables sous forme d'un algorithme.\\

\titre{Exemple} : Savoir à partir d'un couple (algorithme, données initiales) si l'algorithme se termine ou pas n'est pas calculable. \\

\titre{Preuve} : L'ensemble des algorithmes est dénombrable (car un algorithme est un mot fini sur un alphabet fini), tandis que les fonctions ne sont pas dénombrable (théorème de Cantor). \\

\titre{Théorème de Cantor} : Il n'existe pas de bijection entre $E$ et $P(E)$.\\

\titre{Preuve} : \begin{enumerate}
	\item Il existe une injection triviale de $E$ dans $P(E)$
	\item Supposons qu'il existe une bijection $f$ de $E$ dans $P(E)$ et considérons $D=\{x\in E \tq x\not\in P(E)\}$. \\
Comme $f$ est bijective, $\exist y\in E\tq f(y)=D$ \\
Si $y\in D$ alors $y\not\in f(y) = D$, Si $y\not\in D=f(y)$ alors $y\in D$\\
Contradiction
\end{enumerate}
