Un \titre{chemin} de $G=(X,\Gamma)$ est une \titre{suite de sommets} $x_{i_1},x_{i_2},\ldots,x_{i_k}$ telle que $(x_{i_1},x_{i_2})\in\Gamma \ldots (x_{i_{k-1}},x_{i_k})\in\Gamma$. \\

\par

$x_{i_1}=$ \titre{origine} = \titre{ascendant} de $x_{i_k}$ \\
$x_{i_p},1<p<k =$ \titre{sommets intermédiaires}\\
$x_{i_k}=$ \titre{extrémité} = \titre{descendant} de $x_{i_1}$.\\

\par

\titre{notation} : $u=[x_{i_1}*\ldots *x_{i_k}] = [x_{i_1}*x_{i_k}]$. \\
\titre{longueur} : $|u|$ = nombre d'arcs. \\
\titre{concaténation} : $u.v$\\
\titre{propriété} : $|u.v|=|u|+|v[$\\
\titre{circuit} : origine = extrémité. \\
\titre{chemin élémentaire} : $ \forall p,q x_{i_p}=x_{i_q}$
 
