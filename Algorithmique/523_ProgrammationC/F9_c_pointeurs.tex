\titre{Afficher l'adresse d'une variable} : printf("\%p",\&mavariable) \\

\par

\titre{Créer un pointeur} (variable contenant l'adresse d'une autre variable) \\
type *monpointeur = NULL; \\
monpointeur = \&mavariable; \\

\par
\titre{Accéder à une variable via un pointeur} \\
printf("\%d",monpointeur) -> adresse de mavariable \\
printf("\%d",*monpointeur) -> valeur de mavariable \\

\par
\titre{Cas d'utilisation} \\
Envoyer un pointeur à une fonction permet de l'autoriser à modifier ponctuellement la valeur d'une variable sans pour autant la rendre globale. On envoie à la fonction l'adresse d'une variable à modifier à cet instant t.\\

\par

\titre{Différences tableau/pointeur :} Un tableau est un pointeur constant (on ne peut pas changer l'adresse pointée par un tableau. \\

\par

\titre{Type void* : } En général, on donne un type au pointeur pour connaître la "taille" d'une case en mémoire (en fait pointeur = adresse + taille). Avec un type void*, le type n'est pas imposé mais l'indirection est impossible sans typecast.

\newpage
\titre{exemple : memcopy}
