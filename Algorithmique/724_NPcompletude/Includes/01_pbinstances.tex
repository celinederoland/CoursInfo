\titre{Qu'est-ce qu'un problème ?} C'est une question paramétrée dans un ensemble infini. \\

\titre{Instance :} Une instance d'un problème est obtenue en spécifiant la valeur du paramètre.\\

\titre{Résolution :} Un algorithme résoud un problème s'il trouve la solution à toutes ses instances.\\ 

\titre{Heuristique :} Ressemble à un algorithme mais ne trouve pas toujours la même solution. \\

\titre{Encodage d'une instance :} On représente les instances d'un problème par une suite de symboles répondant à un schéma d'encodage.\\

\titre{Taille d'une instance :} Nombre de symboles utilisés pour la représenter.\\

\titre{Remarque :} En général, on ne s'intéresse pas au schéma d'encodage précisément mais à un critère proportionnel à la longueur d'un schéma d'encodage raisonnable.\\

\titre{Complexité temporelle d'un algorithme :} \\$f\ffonc{n}{}$ nb\_max d'étapes élémentaires pour résoudre une instance de taille $n$
