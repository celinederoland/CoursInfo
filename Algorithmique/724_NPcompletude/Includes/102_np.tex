\titre{Voyageur de commerce :} 
\begin{itemize}
	\item E : Trouver une tournée la moins longue possible
	\item O : Quelle est la distance minimale d'une tournée ?
	\item D : Existe-t-il une tournée dont la distance est $\leq k$ ?
\end{itemize}
Montrons l'équivalence entre ces 3 versions du problèmes :
\begin{itemize}
	\item E $\impl$ O et D : trivial
	\item D $\impl$ O : recherche de la distance minimale par dichotomie
	\item O $\impl$ E : admis
\end{itemize}

\titre{Cycles hamiltonnien :} 
\begin{itemize}
	\item E : Trouver un cycle hamiltonnien dans $G$
	\item D : Existe-t-il un cycle hamiltonnien dans $G$ ?
\end{itemize}
Montrons l'équivalence entre ces 2 versions du problème :
\begin{itemize}
	\item E $\impl$ D : trivial
	\item D $\impl$ E : si le graphe n'est pas hamiltonnien, on répond impossible, sinon on enlève les arêtes tant que le graphe reste hamiltonnien.
\end{itemize}
\newpage

\titre{Algos polynomialement équivalents :}
\begin{itemize}
	\item Clique : Existe-t-il une clique de taille $k$ dans $G$ ?
	\item Indé : Existe-t-il dans $G$ un sous ensemble de sommets de taille $k$ sans arêtes communes ?
	\item Couv : Existe-t-il dans $G$ un sous ensemble de sommets de taille $k$ tel que toute arête est adjacente à un de ces sommets ?
\end{itemize}
Montrons l'équivalence polynomiale entre ces 3 problèmes :
\begin{itemize}
	\item Indé $\leq_p$ Clique : $R \ffonc{G,k}{\bar{G},k}$
	\item Clique $\leq_p$ Indé : $R \ffonc{G,k}{\bar{G},k}$
	\item Couv $\leq_p$ Indé : $R \ffonc{G,k}{G,n-k}$
	\item Indé $\leq_p$ Couv : $R \ffonc{G,k}{G,n-k}$
\end{itemize}

\titre{Voyageur de commerce et cycle hamiltonnien :}\\
On part d'un graphe $G$, on met un poids de 1 sur toutes les arêtes.\\
On ajoute les arêtes manquantes pour le rendre complet, avec un poids de 2. \\
On pose $k = $ nombre de sommets de $G$. \\
L'algo $R$ ainsi défini transforme toute instance de Ham en une instance de Voyageur ayant la même réponse.
