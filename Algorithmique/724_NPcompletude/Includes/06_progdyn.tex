\titre{Sous structure optimale :} Une solution à une instance contient en elle la solution de plusieurs sous problèmes. \\

\titre{Mémoriser } les solutions aux sous problèmes pour pouvoir les réutiliser dans le calcul de la solution. \\

\titre{Cas d'utilisation :} La programmation dynamique est souvent utile lorsqu'on a une équation de la forme $T(n) \geq aT(n-b) + f(n)$ \\

\titre{Théorème :} Soit $T(n)$ une fonction de complexité temporelle qui satisfait l'équation : \\ $T(n) = aT(n-b) + f(n)$, avec $a\geq 2, b\geq 1, f(n) \in \Omega(1)$. \\ Alors $\exist c > 1 \tq T(n) \in \Omega(c^n)$\\

\titre{Preuve :} On pose $c = \displaystyle{^b\sqrt{a}}$. Montrons par récurrence que $T(n) \geq c^n-a$. \\
Initialisation : Si $n\leq b$ alors : $c^n - a = c^n - c^b \geq 0$ (car $c>1$ et $b \geq n$). Or $T$ est une fonction de complexité donc $T(n)\geq 0$ \\
Hérédité : Si $n>b$, on suppose que $T(n-b) \geq c^{n-b}-a$. \\
On a alors $T(n) = aT(n-b) + f(n) \geq a(c^{n-b} - a) + f(n) = c^n - a^2 + f(n) \geq c^n - a^2$ (car $f(n) \geq 0$) donc $T(n) \in \Omega(c^n)$\\
\newpage
\titre{Exemple : Nombres de Fibonacci} \\
$\mathrm{Fibo}(n) = \left\{ \begin{array}{ll} 0 \; \mathrm{si} \; n=0 \\ 2 \; \mathrm{si} \; n=2 \\ f(n-1)+f(n-2) \; \mathrm{sinon} \\ \end{array} \right.$\\

\titre{Algo : Découpe de barres} \\ On dispose d'une barre de longueur $n$, et d'un tableau prix indicé de $1$ à $k$. On cherche la manière de découper la barre de façon à maximiser le prix de vente (somme des prix des sous barres). \\

\titre{Algo : Distance de Levenstein} \\
On se dote de 3 opérations sur les mots : 
\begin{enumerate}
	\item Enlever une lettre
	\item Ajouter une lettre
	\item Remplacer une lettre par une autre
\end{enumerate}
Etant donnés deux mots $u$ et $v$ on cherche le nombre minimum d'opérations élémentaires pour transformer $u$ en $v$. \\

\titre{Algo : Impression équilibrée d'un texte} \\
On veut imprimer $n$ mots sur maximum $n$ lignes de taille $N$. Les espaces donnent des pénalités. \\
