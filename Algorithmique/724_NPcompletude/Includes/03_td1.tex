\titre{Question 1}
\begin{enumerate}
	\item $5n^4+10n^2+n-100 = O(n^4)$
	\item $3\log_4(n^5)+6 = O(\ln n)$
	\item $n(\log_{10}n +10) = O(n\ln n)$
	\item $\frac{3n}{n+1} = O(1)$
	\item $\sqrt{n} + \left( \frac{1000}{1001} \right)^n = O(\sqrt{n})$
	\item $n^10 + \left( \frac{1001}{1000} \right)^n = O(\left( \frac{1001}{1000} \right)^n)$
	\item $(4n^3+2n^2+15)(25n^4+10n+1) = O(n^7)$
	\item $2n^2 +2^{n+3} = O(2^n)$
	\item $\frac{2^n}{3^n} = O(\left(\frac{2}{3}\right)^n) = O(1)$
	\item $\frac{3^n}{2^n} = O(\left(\frac{3}{2}\right)^n)$
	\item $2^{\log_2n} + \log_2(2^n) = 0(n)$ 
\end{enumerate}

\titre{Question 2} Tri par ordre croissant sur l'ordre induit par $O$
\begin{enumerate}
	\item $n^100$
	\item $(2^n)^2$
	\item $n!$
	\item $n^n$
	\item $(2^{n^2})$
	\item $A(n,n)$ définie par $\left\{ \begin{array}{l} A(0,n) = n+1 \\ A(m,0) = A(m-1,1) \\ A(m,n) = A(m-1,A(m,n-1)) \end{array}\right.$
\end{enumerate}

\titre{Question 3} 
\begin{enumerate}
	\item $T(n) = 2T(\frac{n}{2}) + n^2$
	\item $T(n) = 4T(\frac{n}{4}) + \sqrt(n)$
\end{enumerate}
On va devoir utiliser le théorème :\\
\titre{Complexité des algos "diviser pour régner" :} Cas $T(n) = aT(\frac{n}{b}) + f(n)$ avec $a \geq 1$ , $b > 1$ et $T(0) = \theta(1) (=1)$ (on peut remplacer $\frac{n}{b}$ par sa partie entière inférieure ou supérieure) \\

\titre{Propriété :} 
\begin{itemize}
	\item Si $f(n) = O(n^{\mathrm{log}_ba-\varepsilon})$ ($\varepsilon > 0)$ alors $T(n)\in \theta(n^{\mathrm{log}_ba})$
	\item Si $f(n) = \theta(n^{\mathrm{log}_ba})$ alors $T(n) = \theta(n^{\mathrm{log}_ba}\ln n)$
	\item Si $f(n) = \Omega(n^{\mathrm{log}_ba+\varepsilon})$ ($\varepsilon > 0$) et $af(\frac{n}{b}) \geq cf(n)$ pour $n$ assez grand et $c$ une constante $< 1$. Alors $T(n) \in \theta(f(n))$
\end{itemize}

J'applique ce théorème sur les deux cas : \\
\begin{enumerate}
	\item $f(n) = n^2$, $n^{\log_ba-\varepsilon} = n^{\log_22 - \varepsilon} = n^{- \varepsilon}$. C'est le 3eme cas donc $T(n) = O(n^2)$
	\item $T(n) = 4T(\frac{n}{2}) + \sqrt(n)$, $n^{\log_ba-\varepsilon} = n^{\log_24 - \varepsilon} = n^{2 - \varepsilon}$. C'est le 2eme cas donc $T(n) = O(n^{\log_24}\ln n) = O(n^2\ln n)$
\end{enumerate}
Le premier cas est préférable.\\

\titre{Question 4}
\begin{enumerate}
	\item $T(n) = T(\frac{n}{2}) + \Omega(\sqrt{n})$. $a=1, b=2, \log_ab = \ln 2 > \frac{1}{2}$
	\item $T(n) = $
\end{enumerate}

\titre{Question 5}
\begin{enumerate}
	\item $O(n)$
	\item Multiplication d'un chiffre par un nombre : temps constant (en binaire), puis toutes les additions font $O(n^2)$
	\item Il faut calculer $ac$, $ad + bc$ et $bd$ puis additionner les résultats. On divise en temps constant en 4 sous problèmes de taille $\frac{n}{2}$, puis combiner est 3 additions donc en $O(n)$ donc $T(n) = 4T(\frac{n}{2}) + O(n)$.
	\item $\log_ba = \log_24 = 2 > 1$ donc $f(n) = O(n) << n^2$ donc on est dans le premier cas donc $O(n^2)$
	\item Karatsuba propose une autre solution : $u = ac$, $v = bd$, $w = (a+b)(c+d)$, on transforme le coef 4 en coef 3. Donc on passe en complexité $O(n^{\log_23})$ et $\log_23 \approx 1,58$
\end{enumerate}

