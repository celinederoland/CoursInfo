\titre{Exercice 1 :}
\begin{enumerate}
	\item $\exists n \; \mathrm{tq} \; \forall \; w \in L \; \mathrm{tq} \; \mathrm{taille}(w) \geq n \; \mathrm{tq} \; \exists \; x,y,z \; \mathrm{tq} \; w=xyz \; \mathrm{et} \; \mathrm{taille}(xy) \leq n \; \mathrm{et} \; \forall \; k \; xy^kz \in L$
	\item 
	\begin{enumerate}
		\item 
		\begin{itemize}
			\item Soit $n = 2k \in \N$
			\item Je choisis $w=(ba^n)^2$
			\item Pour un découpage $xyz$ avec $\mathrm{taille}(xy) \leq n$ on aura $y=(b+1)a^p$ avec $p < n$
			\item En choisissant $k=0$ on a $xy^kz = (b+1)a^{n-p}ba^n$. Si il n'y a pas le $b$ il est clair que le mot ne peut pas contenir deux fois le même mot car il n'y a qu'un seul $b$ dans le mot. Si le $b$ y est, le premier mot commence par $b$ donc le deuxième mot aussi, et du coup ils n'ont pas le même nombre de lettres ($n+1-p$ pour le premier, $n+1$ pour le second)
		\end{itemize}
		\item C'est $L(aa(aa)^*)$
	\end{enumerate}
	\item 
	\begin{enumerate}
		\item 
		\begin{itemize}
			\item Soit $n \in \N$
			\item Je choisis $w=a^nb^{n+1}$
			\item Pour un découpage $xyz$ avec $\mathrm{taille}(xy) \leq n$ on aura $y=a^p$ avec $1\leq p<n$
			\item En choisissant $k=3$ on a $xy^kz = a^{n-p}(a^p)^{3}b^{n+1}$.Il est alors clair qu'il y a plus de $a$ que de $b$ ($n-p+3p=n+2p\geq n+2$. Il y a au moins $n+2$ a et exactement $n+1$ $b$)
		\end{itemize}
		\item 
		\begin{itemize}
			\item Soit $n \in \N$
			\item Je choisis $w=a^nb^{n-1}$
			\item Pour un découpage $xyz$ avec $\mathrm{taille}(xy) \leq n$ on aura $y=a^p$ avec $1\leq p<n$
			\item En choisissant $k=0$ on a $xy^kz = a^{n-p}b^{n-1}$.Il est alors clair qu'il y a moins de $a$ que de $b$ ($p \geq 1$ donc $n-p\leq n-1$ donc $n-p\not>n-1$)
		\end{itemize}
	\end{enumerate}
	\item 
	\begin{enumerate}
		\item 
		\begin{itemize}
			\item Soit $n \in \N$
			\item Je choisis $w=a^{n\times n +2n +1}$
			\item Pour un découpage $xyz$ avec $\mathrm{taille}(xy) \leq n$ on aura $y=a^p$ avec $1\leq p<n$
			\item En choisissant $k=0$ on a $xy^kz = a^{n\times n +2n +1-p}$. Comme $1\leq p<n$, $n^2 < n\times n +2n +1-p < (n+1)^2$
		\end{itemize}
		\item 
		\begin{itemize}
			\item Soit $n \in \N$
			\item Je choisis $w=a^{2^{n+1}}$
			\item Pour un découpage $xyz$ avec $\mathrm{taille}(xy) \leq n$ on aura $y=a^p$ avec $1\leq p<n<2^n$
			\item En choisissant $k=0$ on a $xy^kz = a^{2^{n+1}-p}$. Comme $1\leq p<n$, $2^n < 2^{n+1}-p < 2^{n+1}$
		\end{itemize}
		\item 
		\begin{itemize}
			\item Soit $n \in \N$
			\item Je choisis $w=a^m$ avec $m$ premier supérieur strict à $2n$
			\item Pour un découpage $xyz$ avec $\mathrm{taille}(xy) \leq n$ on aura $y=a^p$ avec $1\leq p<m$
			\item En choisissant $k=m+1$ on a $xy^kz = a^{m(p+1)}$, et $m(p+1)$ n'est pas premier.
		\end{itemize}
	\end{enumerate}
	\item 
\end{enumerate}

\titre{Exercice 2}
\begin{enumerate}
	\item 
	\item Son complémentaire n'est pas régulier
	\item Si le langage $L$ des mots bien parenthésés était régulier, alors $L\bigcup L((^*)^*)$ serait régulier donc le langage des $(^n)^n$ serait régulier.
	\item Il suffit de remplacer tous les $s$ de l'expression régulière par $w$
	\item Si $L$ était régulier alors en remplaçant les symboles sauf les parenthèses par $\varepsilon$ on aurait que le langage des mots bien parenthésés serait régulier.
\end{enumerate}

\titre{Exercice 3}
\begin{enumerate}
	\item 
\end{enumerate}
