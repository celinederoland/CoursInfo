\titre{Fonction $f(X)$} : $\left\{ \begin{array}{l} 
	\mathrm{si} \; \alpha(X) \; \mathrm{alors} \; \mathrm{val} \\
	\mathrm{sinon} \; f(X') \end{array} \right.$ \\

\titre{Simplification :} Pour analyser le temps d'exécution d'une fonction récursive, on va essayer de résumer à l'aide d'un seul entier $n$ le contexte $X$ de façon à ce que $n$ (correspondant généralement à la taille associée au contexte) décroit à chaque appel de $f$. \\

\titre{Exemple 1 :} \\
\code{
	Fonction F(m : entier) : entier $\theta(1)$\\
	Debut\\
		Si m = 0 $\theta(1)$ alors retourner $\vide$ $\theta(1)$\\
		Sinon retourner m*F(m-1) $\theta(1) + T(m - 1)$\\
	Fin\\
}

En pire cas : 
\begin{itemize}
	\item $T(0) = \theta(1)$
	\item $T(n) = \theta(1) + T(n-1)$
	\item Suite arithmétique de premier terme 1 et de raison 1 donc $T(n) = \theta(n+1) = \theta(n)$
	\item Donc factorielle s'exécute en temps linéaire
\end{itemize}

\newpage

\titre{Exemple 2 :} \\
\code{
	Fonction Max(V:Tableau[] de Elem, a,b : entier) : Elem $\theta(1)$\\
	Debut\\
		Si a = b alors retourner V[a] $\theta(1)$\\
		Sinon retourner Max2(V[a],Max(V,a+1;b)) $\theta(1) + T(n-1)$\\
	Fin\\
}

En pire cas :
\begin{itemize}
	\item $n = b - a$ (car $b$ et $a$ désignent les indices du tableau entre lesquels on cherche le max)
	\item Question : cout du passage de paramètre (par adresse : constant, par valeur : $n$) 
	\item $T(0) = \theta(1)$
	\item $T(n) = \theta(1) + \theta(n-1)$
	\item Suite arithmétique de premier terme 1 et de raison 1 donc $T(n) = \theta(n+1) = \theta(n)$
	\item Donc Max s'exécute en temps linéaire
\end{itemize}

\titre{Exemple 3 :} Recherche dichotomique récursive \\
\code{
Fonction Dicho(V:Tab[]; a,b:entier; x: Elem) :booléen $\theta(1)$\\
Var m: entier\\
Debut \\
	Si a = b alors retourner (x=V[a]) $\theta(1)$\\
	Sinon \\
		m $\leftarrow$ $\frac{a + b}{2}$ \\
		Si x $\leq$ V[m] alors retourner Dicho(V,a,m,x) $T(\frac{n}{2})$\\
		Sinon retourner Dicho(V,m+1,b,x) $T(\frac{n}{2})$\\
		FinSi\\
	FinSi\\
Fin\\
}

En pire cas :
\begin{itemize}
	\item $n = b-a$
	\item $T(0) = \theta(1)$
	\item $T(n) = \theta(1) + T(\frac{n}{2})$
	\item Donc $T(n) = \theta(\ln n)$
\end{itemize}

\titre{Exemple 4 :} Fibonacci \\
\code{
Foncion Fib(m:entier) :entier \\
Debut\\
	Si m = 0 alors retourner 0 \\
	Sinon si m = 1 alors retourner 1 \\
	Sinon retourner Fib(m - 1) + Fib(m - 2) \\
	Fin Si\\
Fin\\
}

En pire cas :
\begin{itemize}
	\item Le nombre d'appels à Fib est exactement le résultat Fib(m) + ou - 1
	\item Avec utilisation de l'exemple : Fib(m) $<<$ Myst(m)
	\item $T(0) = T(1) = \theta(1)$
	\item $T(n) = \theta(1) + T(n-1) + T(n-2)$
	\item Or Fib(n)/Fib(n-1) tend vers le nombre d'or $\phi$
\end{itemize}

\titre{Exemple intermédiaire :}
\code{
Fonction myst(m:entier):entier \\
Debut \\
	Si m = 0 retourner 1\\
	Sinon retourner myst(m-1) + myst(m-1) \\
	FinSi \\
Fin\\
}

\begin{itemize}
	\item $T(0) = \theta(1)$ 
	\item $T(n) = \theta(1) + 2T(n-1)$
	\item $T(n) = 1 + 2 + 2^2 + \ldots + 2^n = 2^{n+1} - 1 = \theta(2^n)$
\end{itemize}




