\titre{Complexité des algos "diviser pour régner" :} Cas $T(n) = aT(\frac{n}{b}) + f(n)$ avec $a \geq 1$ , $b > 1$ et $T(0) = \theta(1) (=1)$ (on peut remplacer $\frac{n}{b}$ par sa partie entière inférieure ou supérieure) \\

\titre{Propriété :} 
\begin{itemize}
	\item Si $f(n) = O(n^{\mathrm{log}_ba-\varepsilon})$ ($\varepsilon > 0)$ alors $T(n)\in \theta(n^{\mathrm{log}_ba})$
	\item Si $f(n) = \theta(n^{\mathrm{log}_ba})$ alors $T(n) = \theta(n^{\mathrm{log}_ba}\ln n)$
	\item Si $f(n) = \Omega(n^{\mathrm{log}_ba+\varepsilon})$ ($\varepsilon > 0$) et $af(\frac{n}{b}) \geq cf(n)$ pour $n$ assez grand et $c$ une constante $< 1$. Alors $T(n) \in \theta(f(n))$
\end{itemize}

\titre{Exemple 7 :} La recherche dichotomique
\begin{itemize}
	\item $T(n) = T(\frac{n}{2}) + 1$ 
	\item On est dans le deuxième cas, donc $T(n) = \theta(\ln n)$
\end{itemize}

\titre{Exemple 8 :} Tri fusion \\
\code{
Fonction TriFusion(V:Tab,n:entier)\\
Var $n_1,n_2$ : entier \\
Var $V_1,V_2$ : Tab \\
Debut \\
	Si $n \geq 2$ Alors \\
		Diviser$(V,n,V_1,n_1,V_2,n_2)$ ($\theta(n)$)\\
		TriFusion$(V_1,n_1)$\\
		TriFusion$(V_2,n_2)$\\
		Fusionner$(V_1,n_1,V_2,n_2,V)$ ($\theta(n)$)\\
	FinSi\\
Fin \\
}

Temps d'exécution :
\begin{itemize}
	\item $T(n) = 2T(\frac{n}{2}) + \theta(n)$
	\item $f(n) = \theta(n)$ donc on est dans le deuxième cas
	\item $T(n) = \theta(n\ln n)$
\end{itemize}
