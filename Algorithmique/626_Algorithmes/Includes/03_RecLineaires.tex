\titre{Récurrence linéaire :} $$T(n+ p + 1) = f(n) + \displaystyle{\sum_{k=0}^{p}} a_kT(n+k)$$
	$$T(n+1) = aT(n) + 1$$
	$$T(n+2) = aT(n+1) + bT(n) + 1$$ \\

\titre{Résolution du cas d'ordre 1 :} \\

	$T(n) = \left\{ \begin{array}{c}
		n + 1 = \theta(n)\; \mathrm{si} \; a = 1 \\
		\frac{a^{n+1} - 1}{a - 1} = \theta(a^n) \\
	\end{array} \right.$ \\

\titre{Exemple 5 :} Hanoï \\
\code{
Fonction Hanoi(n,i,j,k)\\
Debut \\
	Si n > 0 alors \\
		Hanoi(n-1,i,j,k) \\
		Affiche("Je déplace de i vers j")\\
		Hanoi(n-1,k,j,i) \\
	FinSi \\
Fin \\
}

Temps de calcul :
\begin{itemize}
	\item $T(0) = \theta(1)$
	\item $T(n) = \theta(1) + 2T(n-1)$
	\item Donc $T(n) = \theta(2^n)$
\end{itemize}

\titre{Résolution du cas d'ordre 2 :} On simplifie le problème en faisant abstraction du 1 : On considère $T(n+2) = aT(n+1) + bT(n)$ ie $T(n+2) - aT(n+1) - bT(n) = 0$. \\
On pose $P(X) = X' - aX - b$. Il a deux racines $r_1$ et $r_2$ et la solution s'écrit sous la forme $\alpha r_1^n + \beta r_2^n$ (il suffit de le vérifier). \\
$$T(n) = \theta(r_1^n)$$

\titre{Fin du cas d'ordre 2 :} Le 1 ajoute une fois Fibonacci à la fin, donc ne change pas le $\theta$ \\

\titre{Fin de l'exemple 4 :} Fibonacci 
\begin{itemize}
	\item $P(X) = X^2 - X - 1$ donc $r_1 = \frac{1 + \sqrt{5}}{2} = \phi$ et $r_2 = \frac{1 - \sqrt{5}}{2}$ donc $r_2^n$ tend vers 0.
	\item $T(n) = \theta(\phi^n)$
\end{itemize}

\titre{Exemple 6 :} \\
\code{
Fonction Dummy(n:entier) :entier \\
	Si n $\leq$ 3 retourner 1 \\
	Sinon retourner Dummy(n-2) + Dummy(n-4) \\
	Fin Si \\
Fin \\
}

Solution :
\begin{itemize} 
	\item Le polynome est de la forme $X^4 - X^2 - 1$
\end{itemize}
